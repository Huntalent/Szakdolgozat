\Chapter{Összefoglalás}

A szakdolgozatom pénzügyi adatok elemzéséhez használható interaktív megjelenítő weboldalt mutatott be elméleti és gyakorlati példákkal egyetemben, valamint a hozzá elkészített alkalmazást. A dolgozat felépítése jellemzően az elméleti tudástárból a gyakorlati megvalósítás felé terjedt. Az elméleti részben feltártam a Grafikonrajzoló lapon elérhető diagrammok típusait, ismertettem tőzsdei alapfogalmakat, illetve kifejtettem az oldalon található kulcsszavakat, amelyek tudása elengedhetetlen. Ezek után az általam írt program tervezésének lépéseit mutattam be, majd a hozzá felhasznált technológiákat részleteztem.

	Az alkalmazás fejlesztése Visual Studio Code környezetben történt, munkámat elősegítette a Live Server bővítmény által kínált kényelmes élő szerver szolgáltatás. A szerverhez használt kódokat Node.js nyelven írtam meg és az adatokat az EOD Historical Data nevezetű webhely szolgáltatta. Mivel az portálon fellelhető teljes adatkészlet nem elérhető ingyenes verzióban, ezért ezt a problémát úgy abszolváltam, hogy Postman alkalmazás segítségével lekértem a hat befeketetésem adatait és külön json fájlban mentettem el őket. Az alkalamzás ezen adatok 1 éves halmazával tud operálni. 

	A klienst JavaScript és jQuery nyelveken vittem véghez, emellett HTML elemekkel építettem fel a sémáját, továbbá CSS és Bootstrap komponensekkel vittem véghez a megjelenésének kialakítását. A gráfok szerkezetében a Chart.js előre definiált könyvtárát vettem igénybe és úgy alakítottam, hogy prominens legyen a látogató számára. Úgy gondoltam, hogy ne kelljen mindig végigkattintani a diagrammok adatait, ezáltal a Grafikonrajzoló három opciójának alapértelmezett választásként adtam meg, amiket természetesen lehet kombinálni a felhasználó igénye szerint. A grafikon megjelenésének nélkülözhetetlen részei a kezdeti és záró dátumok, ennek okán csakis akkor jelenik meg ha a vizsgálati feltételeknek eleget tévő paraméterek kerülnek kijelölésre. Ehhez szükséges a szerver és kliens kommunikációja amit API hívásokon keresztül valósítottam meg.

	A kész szoftver legnagyobb előnye, hogy önálló weboldalként is képes megállni a helyét. Struktárja modern webfejlesztési módszerek szerint készült, tehát könnyen kezelhető, gyorsan működik és interaktív.A program a grafikonok változatos megjelenítésével, emellett többféle árfolyam lehetőségével szándékozik kiemelkedni a többi hasonló pénzügyi adatokat elemző alkalmazások közül. 

	Véleményem szerint sikerült egy működő weboldalt létrehoznom, amihez néhány továbbfejlesztési lehetőség ötletével álltam elő.

\section{Továbbfejlesztési lehetőségek}

A szakdolgozatom készítése alatt rengeteg ötlet és koncepció jutott eszembe, de mivel a rendelkezésre álló idő végleges, így korlátoznom kellett magam a fontosabb funkciók megalkotására. Az első és talán legszembetűnőbb opció ami nincs a szoftveremnél az adatbázis. A Tervezési fázisnál még szerepelt, de később elvetettem az ötletet, mert nem éreztem feltétlenül szükségesnek. Döntésemet az is indokolta, mint ahogy a dolgozatom címében is szerepel "Interaktív megjelenítő eszköz", így az energiaforrást a grafikonrajzolóra fordítottam. Viszont ha mondjuk a tanulmányaim során megismert MySQL adatbázist bevezettem volna, akkor az ötödik fejezetben említett hiányzó funkciók közül kettőt kipipálhatok.

	Amivel még lehet emelni az oldal színvonlát, ha több befektetés áll rendelkezésre, és egyszerre nem csak egy alapot lehet kiválasztani. Ezen funkció nagyon átgondolt struktúrát igényel, mert ha több alapot is ki lehet választani, akkor mindegyik megjelenítési típusnál le kell tesztelni miként működik. Hiszen ami mondjuk működőképes egy vonaldiagramnál, az nem biztos, hogy hasonlóképpen működne a kördiagramnál. Tehát alaposan kell tanulmányozni a diagrammok típusait, hogy milyen szabályosságok érvényesek a különféle típusokra és a kellő háttérinformáció megszerzése után úgy kellene megtervezni a rendszert, hogy minden elemre működjön. Talán ezért is hiányzik a rangosabb oldalakról a többféle típus kiválasztása.

	Amit még a kutatásaim során néhány oldalon felfedeztem, az a bejelentkezés és profil részleg kialakítása. A szubjektív véleményem, hogy nem tartom túl szükséges elemnek egy ilyesfajta weboldalon, de példának okáért egy alternatívaként arra hasznos lehet, hogy a kedvenc részvényeket nyomon lehessen követni.

\section{Summary}

My thesis presented an interactive display website that can be used for analyzing financial data, together with theoretical and practical examples, as well as the application prepared for it. 
The structure of the thesis typically extended from the theoretical knowledge base to the practical implementation. 
In the theoretical part, I explored the types of diagrams available on the Graph Drawing tab, explained basic stock concepts, and explained the keywords on the page, of which knowledge is essential. After that, I presented the steps of designing the program that I wrote, and then detailed the technologies used for it.

	The application was developed in the Visual Studio Code environment, my work was facilitated by the convenient live server service offered by the Live Server extension. The code that used for the server is written by Node.js and the data was provided by the EOD Historical Data website. 
Since the complete data set found on the portal is not available in the free version, I solved this problem by using the Postman application to retrieve the data of my six blackouts and save them in a separate json file. The application can operate with a 1-year set of these data.

	I implemented the client in JavaScript and jQuery languages, in addition, I built its schema with HTML elements, and I implemented the design of its appearance with CSS and Bootstrap components. In the structure of the graphs, I used the predefined library of Chart.js and designed it so that it is prominent for the visitor. I thought that it would not be necessary to always click through the data of the charts, so I gave a default choice for the three options of the Graphic Designer, which of course can be combined according to the user's needs. The start and end dates are indispensable parts of the appearance of the graph, which is why they only appear when parameters that meet the test conditions are selected. This requires server and client communication, which I implemented through API calls.

	The biggest advantage of the ready-made software is that it can also stand its ground as an independent website. Its structure is made according to modern web development methods, so it is easy to use, works quickly and is interactive. The program intends to stand out from other applications that analyze similar financial data with the varied display of graphs and, in addition, the possibility of multiple exchange rates.
