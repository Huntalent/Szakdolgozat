\Chapter{Bevezetés}

	A szakdolgozatom célja egy weboldal megtervezése és implementálása, amely alkalmas többféle pénzügyi adat elemzéséhez. Ehhez különféle befektetési alapokat használok, továbbá egy grafikonrajzolót. Az oldalnak bemutatom a szerkezeti felépítését, elkészítését és az alkalmazás működését.
	A pénzügyi adatok, vagy tőzsdei árfolyamok megismerése lehet nagyon egyszerű, de gyakran nehézséget jelent azon személyek számára, akik előszőr próbálkoznak meg vele a hétköznapokban. Ezért a dolgozatom olyan lehetőségeket mutat be, amelyekkel átláthatóbban lehet megérteni és kezelni őket.

	Ezeket a lehetőségeket kétféle alternatívára osztottam szét. Az egyik opció a részletes leírása és ismertetése a befektetési alapoknak, kiegészítve egy táblázattal, amely ábrázolja a hozam-kockázat profilt, továbbá egy alap által elért éves nettó hozam sávot. Másik opcióként külön oldalon grafikonrajzoló használható. Lehetőség van kezdési- és zárópontot megadni, így különböző intervallumokat kiválasztva tudjuk vizsgálni az adott befektetési alapot. A dolgozat az utóbbira fektet nagyobb hangsúlyt több okból kifolyólag is. Ezt elsősorban az indokolja, hogy az így nyert adatokat rugalmasabban tudja kezelni a felhasználó, mivel egyszerre több kiválasztási lehetőség áll rendelkezésre, amivel részletesebb adathalmaz nyerhető ki. A teljesség igénye nélkül felsorolok pár lehetőséget. Például, napi adatokat lehet vizsgálni olyan kategóriák szerint, mint napi legmagasabb, vagy éppen napi legalacsonyabb árfolyam. Továbbá a legelterjedtebb ábrázolási forma ha vonaldiagrammon ábrázoljuk a kívánt adatokat. Az alkalmazás elkészítése során számomra szempont volt, hogy ettől függetlenül több Választást kínáljak a felhasználónak, így lehetősége nyílik több ábrázolási módszer közül is választani, úgy mint például oszlopdiagram, vagy kördiagram.

	A megjelenítéshez és az algoritmusok fejlesztéséhez több technológia is kiválasztásra került, többek között HTML5 a weboldal vázát, CSS a weboldal megjelenítését írja le, Node.JS amellyel a webhely szervere működik és végezetül JavaScript nyelven válik elérhetővé a grafikonrajzoló teljes egésze. Ezeken felül még használtam további JavaScript könyvtárat, mint a JQuery, illetve az oldal reszponzivitásáért felel a nyílt forráskódú kliens oldali keretrendszer a Bootstrap5.

	Több grafikonrajzoló és eszközkezelő weboldal elérhető a különböző befektetési vállalatoknak az Interneten.  Ebből kifolyólag felmerülhet a kérdés, miért volt szükség még egy weboldal elkészítésére? Többek között erre a kérdésre is választ kaphatunk a szakdolgozat elolvasása után.
